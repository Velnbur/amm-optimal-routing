\documentclass[beamer]{standalone}

\begin{document}

\begin{frame}\frametitle{У чому різниця між АММ та біржами на ордер буках?}
  \begin{block}{Order Book}
    Ордер Буки (з англ. \textit{order book}) --- це список заявок на купівлю та
    продаж активу, що відображається у вигляді таблиці з ціною та об'ємом.
  \end{block}
  \begin{columns}
    \column{0.5\textwidth}
    \begin{table}
      \begin{tabular}{ c | c }
        Об'єм & Ціна \\
        \hline \hline
        1 & 10,0 \\
        \rowcolor{red} 2 & 9,95 \\
        3 & 9,87 \\
        4 & 9,71
      \end{tabular}
      \caption{Таблиця на купівлю}
    \end{table}
    \column{0.5\textwidth}
    \begin{table}
      \begin{tabular}{ c | c }
        Ціна & Об'єм \\
        \hline \hline
        10,1 & 1 \\
        \rowcolor{green} 9,95 & 2 \\
        10,3 & 3 \\
        10,4 & 4
      \end{tabular}
      \caption{Таблиця на продаж}
    \end{table}
  \end{columns}
\end{frame}

\begin{frame}\frametitle{У чому різниця між АММ та біржами на ордер буках?}
  АММ для додатнього вектору вхідних об'ємів $\Delta \mathbf{x} \in \mathbb{R}^{n}$ та
  вихідних $\Delta \mathbf{y} \in \mathbb{R}^{n}$ визначає функцію
  $f(\Delta \mathbf{x}, \Delta \mathbf{y})$ над цими вкладами, що відповідає за визначення
  коректності трейду по правилам AMM.

  У випадку, якщо для даних $\Delta \mathbf{x}$ та $\Delta \mathbf{y}$ трейд вважається
  некоректним, то обмін не стається.
\end{frame}

\begin{frame}\frametitle{У чому різниця між АММ та біржами на ордер буках?}
  \begin{block}{Чим АММ краща за ордер буки?}
    \begin{itemize}
      \item Прозорість
      \item Детермінованість
      \item Простота в імплементації
    \end{itemize}
  \end{block}

  На цій моделі трейдингу побудовано багато децентралізованих бірж, таких як:

  \begin{itemize}
    \item Uniswap
    \item Curve
    \item SushiSwap
  \end{itemize}
\end{frame}

\end{document}
