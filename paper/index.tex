\documentclass[14pt]{extarticle}
% \documentclass[14pt]{article}
% \usepackage[utf8]{inputenc}
% \usepackage[T1,T2A]{fontenc}
\usepackage{fontspec}
\setmainfont{Times New Roman}
% \usepackage{mathptmx}
% \usepackage{txfonts}
\usepackage{setspace}
\onehalfspacing{}
\usepackage{titlesec}
\usepackage{moresize}

\usepackage{subfiles}

\usepackage{csquotes}
% \usepackage{capt-of}
\usepackage[normalem]{ulem}
\usepackage{amsmath}
\usepackage{amsthm}
\usepackage{graphicx}
\usepackage{hyperref}
\usepackage{xcolor}
\usepackage[left=25mm,right=10mm,top=20mm,bottom=20mm]{geometry}
\usepackage{amsfonts}
\usepackage{algpseudocode}
\usepackage{algorithm}
\usepackage{amssymb}
\usepackage{tikz}
\usetikzlibrary{er,positioning}
\usepackage{pgfplots}
\usetikzlibrary{graphs,quotes}


\usepackage{fancyhdr}
\pagestyle{fancy}

% \fancyhf{} % clear all header and footer fields
\fancyhead[RO,RE]{\thepage} %RO=right odd, RE=right even
\fancyhead[L]{}
\fancyfoot{}
\renewcommand{\headrulewidth}{0pt}
\renewcommand{\footrulewidth}{0pt}


\usepackage[ukrainian]{babel}

\usepackage{polyglossia}
\setdefaultlanguage{ukrainian}
\setotherlanguages{english}
\setmonofont{Fira Code}[
  Scale=0.85
]
% \newfontfamily{\cyrillicfonttt}{FiraCode Nerd Font}

\usepackage[format=plain,
	labelfont={bf},
	textfont=it]{caption}

\author{Байбула Кирило Аленович}
\date{\today}

\def\title#1{\gdef\@title{#1}\gdef\THETITLE{#1}}

\title{ЗНАХОДЖЕННЯ ОПТИМАЛЬНОГО ШЛЯХУ ОБМІНУ У БІРЖАХ НА ОСНОВІ МАРКЕТ МЕЙКЕРІВ
  З ФУНКЦІЯМИ ОБМІНУ КОНСТАНТНОГО ДОБУТКУ}

\usepackage{biblatex}
\addbibresource{./paper/index.bib}

\usepackage{totcount}
\newtotcounter{citenum}
\AtEveryBibitem{\stepcounter{citenum}}

\usepackage{zref-totpages}

\newtheorem{theorem}{Теорема}
\newtheorem{lemma}{Лема}
\renewcommand\qedsymbol{$\blacksquare$}

\begin{document}

\begin{titlepage}
	\begin{center}
		\textbf{КИЇВСЬКИЙ НАЦІОНАЛЬНИЙ УНІВЕРСИТЕТ \\ ІМЕНІ ТАРАСА ШЕВЧЕНКA} \\
		Факультет комп'ютерних наук та кібернетики \\
		Кафедра дослiдження операцiй

		\vspace{2.0cm}
		{\large\textbf{Кваліфікаційна робота} } \\
		\textbf{на здобуття ступення бакалавра} \\
		за спеціальністю 113 Прикладна математика \\
		на тему: \\
		\textbf{\THETITLE}
		\vspace{0.6cm}
		% \begin{flushleft}
		% 	{\small  \\
		% 		 \\
		% 		\vspace{0.5cm}
		% 		Науковий керівник: \\
		% 		 \\
		% 		 \\
		% 		}
		% \end{flushleft}

		\renewcommand{\arraystretch}{0.8} % Default value: 1
		\begin{tabular}{l@{\hspace{5cm}} c}
			{ Виконав студент 4-го курсу}            &                          \\
			{ Байбула Кирило Аленович}               & \underline{\hspace{3cm}} \\
			                                         & {\small (підпис)}        \\
			{ Науковий керівник}                     &                          \\
			{ доктор фіз.-мат. наук, доцент,}        &                          \\
			{ професор кафедри дослідження операцій} &                          \\
			{ Самойленко Ігор Валерійович}           & \underline{\hspace{3cm}} \\
			                                         & {\small (підпис)}        \\
		\end{tabular}

		\begin{tabular}{c@{\hspace{8cm}} l@{}}
			 & { Засвідчую, що в цій роботі немає запозичень}           \\
			 & { з праць інших авторів без відповідних}                 \\
			 & { посилань.}                                             \\
			 & \begin{tabular}{l@{\hspace{4cm}} c}
				   {Студент} & \underline{\hspace{3cm}} \\
				             & {\small (підпис)}        \\
			   \end{tabular}          \\
			 & { Роботу розглянуто й допущено до захисту }              \\
			 & { на засіданні кафедри дослідження операцій }            \\
			 & { ``31'' травня 2024р., }                                \\
			 & { протокол №13 }                                         \\
			 & \begin{tabular}{l@{\hspace{2cm}} c}
				   {Завідувач кафедри} &                          \\
				   {О. В. Іксанов}     & \underline{\hspace{3cm}} \\
				                       & {\small (підпис)}        \\
			   \end{tabular} \\
		\end{tabular}

		\vfill

		\vspace{0.8cm}
		Київ ---  2024
	\end{center}
\end{titlepage}
\newpage


\section*{РЕФЕРАТ}\label{sec:abstract}

Обсяг роботи~\ztotpages{} сторінки та \total{citenum} використаних джерел.

БІРЖІ, МАРКЕТ МЕЙКЕРИ, АММ, КОНСТАНТНІ ФУНКЦІЇ ОБМІНУ, ДИНАМІЧНІ ГРАФИ,
ОПТИМІЗАЦІЯ, АЛГОРИТМИ, ПРОТОКОЛИ.

Дана робота присвячена дослідженню методів обробки бірж основаних на методі
автоматизованих маркет-мейкерів (з англ. ``Automated Market Maker'') AMM і їх
часткового випадку із константними функціями обміну. Головною метою роботи є
розробка та імплементація алгоритму знаходження найоптимальнішого шляху для
динамічного графа подібної біржі. Оптимальним буде вважатися шлях що між всіма
можливими шляхами пари валют $X/Y$ буде отримувати найбільшу кількість $Y$ за
$X$.

\newpage

\renewcommand{\contentsname}{ЗМІСТ}
\setcounter{tocdepth}{2}
\tableofcontents
\newpage

\section*{СКОРОЧЕННЯ ТА УМОВНІ ПОЗНАЧЕННЯ}\label{sec:notation}

\begin{center}
	\begin{tabular}{ll}
		Позначення       & Значення                                               \\[0pt]
		\hline{}
		АММ              & Біржі основані на методі автоматичних маркет майкерів. \\[0pt]
		ММКФ             & Маркет мейкери із константними функціями               \\[0pt]
		ММКД             & Маркет мейкери із константним добутком                 \\[0pt]
		\(X/Y\)          & Пара на біржі валюти типу \(X\) та \(Y\)               \\[0pt]
		$base$           & Позначення базової валюти біржі                        \\[0pt]
		\(X \implies Y\) & Короткий запис обміну валюти \(X\) на \(Y\)            \\[0pt]
	\end{tabular}
\end{center}

\newpage

\section{ВСТУП}\label{sec:intro}
\subfile{./paper/sections/1_0_intro}

\newpage

\section{Функція обміну}\label{sec:swap-function}
\subfile{./paper/sections/2_0_swap_function}

\newpage

\section{Формалізація графа}\label{sec:graph}
\subfile{./paper/sections/3_0_graph}

\newpage

\section{Імплементація алгоритму}\label{sec:algorithm-impl}
\subfile{./paper/sections/4_0_algorithm_impl}

\newpage

\section{ВИСНОВКИ}

У даній роботі було розглянуто питання оптимізації обміну валют на біржах
основаних на методі автоматизованих маркет-мейкерів. Було розглянуто питання
формалізації графа біржі та розроблено алгоритм знаходження найоптимальнішого
шляху обміну валют із переведенням задачі до стандартної задачі пошуку

\newpage

\printbibliography{}
\newpage

\section{ДОДАТОК А}\label{sec:addition-a}

У цьому додатку наведений приклад SQL коду для оформлення бази даних у
імплементації даного алгоритму.

Таблиця блоків \texttt{blocks}:

\begin{verbatim}
CREATE TABLE blocks (
    height BIGINT   PRIMARY KEY,
    hash   CHAR(66) NOT NULL UNIQUE
);
\end{verbatim}

Таблиця токенів \texttt{tokens}:

\begin{verbatim}
CREATE TABLE tokens (
    id SERIAL PRIMARY KEY,

    address CHAR(42) NOT NULL UNIQUE,

    name   TEXT NOT NULL,
    symbol TEXT NOT NULL,
);
\end{verbatim}

Таблиця пар \texttt{pairs}:

\begin{verbatim}
CREATE TABLE IF NOT EXISTS pairs (
    id      SERIAL   PRIMARY KEY,
    address CHAR(42) NOT NULL UNIQUE,
    token0  INTEGER  NOT NULL,
    token1  INTEGER  NOT NULL,
    block   INTERGER NOT NULL,

    FOREIGN KEY (token0) REFERENCES tokens(id),
    FOREIGN KEY (token1) REFERENCES tokens(id),
    FOREIGN KEY (block)  REFERENCES blocks(height),
);
\end{verbatim}

Таблиця резервів \texttt{reserves}:

\begin{verbatim}
CREATE TABLE IF NOT EXISTS reserves (
    id    SERIAL  PRIMARY KEY,
    pair  INTEGER NOT NULL,
    block INTEGER NOT NULL,

    reserve0 NUMERIC NOT NULL,
    reserve1 NUMERIC NOT NULL,

    FOREIGN KEY (pair)  REFERENCES pairs(id),
    FOREIGN KEY (block) REFERENCES blocks(height)
);
\end{verbatim}


\end{document}
